\addcontentsline{toc}{section}{Załączniki}

\subsubsection*{Załącznik 1 - Kod umożliwiający etykietowanie bazy nagrań.}

Poniżej przedstawiono kod demonstrujący działanie narzędzia do etykietowania.

\definecolor{codegreen}{rgb}{0,0.6,0}
\definecolor{codegray}{rgb}{0.5,0.5,0.5}
\definecolor{codeorange}{rgb}{1,0.49,0}
\definecolor{backcolour}{rgb}{0.95,0.95,0.96}

\lstdefinestyle{mystyle}{
    backgroundcolor=\color{backcolour},
    commentstyle=\color{codegray},
    keywordstyle=\color{codeorange},
    numberstyle=\tiny\color{codegray},
    stringstyle=\color{codegreen},
    basicstyle=\fontsize{10}{10}\selectfont\ttfamily,
    breakatwhitespace=false,
    breaklines=true,
    captionpos=b,
    keepspaces=true,
    numbers=left,
    numbersep=5pt,
    showspaces=false,
    showstringspaces=false,
    showtabs=false,
    tabsize=2,
    xleftmargin=10pt,
}

\lstset{style=mystyle}

\begin{lstlisting}[language=Python, mathescape=true]
# import bibliotek
import os
from pathlib import Path
import random
import soundfile as sf
import sounddevice as sd


with open("labels.txt", "a") as f:
    for idx, filestem in enumerate(files):
        # zapisanie numeru opisywanego pliku 
        # do zmiennej "file_number"
        file_number = int(len(done_files)) + int(idx) + 1
        while True:
            # zapisanie sciezki do opisywanego pliku 
            # do zmiennej "filename"
            filename = (
                f"speech_dataset/{str(filestem)}.flac"
            )

          

\end{lstlisting}