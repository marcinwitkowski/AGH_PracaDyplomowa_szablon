\section{Wyniki}

To najważniejszy rozdział pracy, gdyż udowadnia co zostało zrobione. Powinien więc być tak napisany by opowiadał historię. Każde wyniki powinny być komentowane w kontekście celów pracy. Przykładowo całość retoryki można zbudować jako:

\subsection{Wyniki referencyjne}

Początkowo przeanalizowano system referencyjny składający się z (...). Tabela \ref{tab:wyniki1} przedstawia wyniki otrzymane dla pokoi o czasach pogłosów .... . Dane symulowane otrzymane metodą źródeł pozornych.

\begin{table}[h]
\centering
\caption{Skuteczność analizowanych metod wyrażona w mierze EER dla różnych czasów pogłosów.}
\begin{tabular}{@{}lllll@{}}
\toprule
Czas pogłosu RT60 {[}ms{]}      & 300          & 600          & 900          & 1200         \\ \midrule
Metoda 1                        & 3.5          & 4.5          & 4.9          & 7.3          \\
Algorytm 2                      & 2.5          & 3.3          & 3.9          & 6.8          \\
Zaproponowana metoda 2 & \textbf{1.5} & \textbf{2.7} & \textbf{3.0} & \textbf{4.5} \\ \bottomrule
\end{tabular}
\label{tab:wyniki1}
\end{table}

Wyniki w tabeli \ref{tab:wyniki1} wskazują, że metoda 2 jest lepsza niż metoda 1. Niemniej jednak zaproponowana w pracy metoda 3 umożliwia poprawę skuteczności ..... w warunkach pogłosowcyh.


Tu powinny znaleźć sie opisy otrzymanych wyników. Jeśli przeprowadzono kilka eksperymentów to warto tu opisać jak je osiągnięto. Wyniki powinno się raportować na wykresach, tabelach, histogramach, confussion matrix, scatter plot itp. 

\subsection{Porównanie}
Na koniec tego rozdziału jeśli wyników było sporo warto je podsumować 2-3 akapitami ogólnego komentarza, parafrazując część zdań z tego rozdziału.