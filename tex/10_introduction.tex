\section{Wstęp teoretyczny}

Tu powinien znaleźć się opis kontekstu pracy. Dlaczego realizacja tematu jest ważna? W czym wyniki pracy mogą pomóc? Dobrze w tej sekcji wstawić kilka refenencji. W przypadku stosowania linków można je zamieścić jako przypis na dole strony za pomocą \textit{footnote}. Generalnie jednak lepiej posługiwać się źródłami z renomowanych źródeł takich jak czasopisma IEEE, ważne konferencje, standardy, patenty, normy. Przykład w kolejnym akapicie. 

Fobia ta jest często uważana za jedną z~najpowszechniejszych, dotyka bowiem około 73\% populacji świata\footnote{\url{https://nationalsocialanxietycenter.com/2017/02/20/public-speaking-and-fear-of-brain-freezes/}, (dostęp 10.10.2022) }. Zagadnienie to zbadano na przykład dla studentów w Malezji~\cite{tse2012glossophobia}, czy Hiszpanii~\cite{marques2021glossophobia}.

\subsection{Motywacja}
\label{subsec:motywacja}

Jeśli można bardziej sprecyzować motywację pracy (np udział w projekcie, zawężenie generalnego problemu opisanego wcześniej do konkretnego problemu). Jeśli trudno to zrobić można pominąć tę sekcję.


\subsection{Cele pracy}
\label{subsec:cele_pracy}
Tu nalezy precyzyjnie wypisać cele pracy. Można w punktach, w taki sposób by można było łatwo się odnieść do nich w treści a zwłaszcza w podsumowaniu pracy.

\subsection{Przegląd literatury}
Tło literaturowe i zarysowanie jaki jest stan rzeczy na podstawie publikacji.

Jeśli jest tego sporo tę część można przenieść do osobnego rozdziału (section) 2
