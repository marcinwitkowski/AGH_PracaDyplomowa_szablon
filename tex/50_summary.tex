\section{Podsumowanie}
% [MWI] Osiągnięto cele.

W podsumowaniu należy skomentować czy i jak udało sie zrealizować każdy cel pracy zdefiniowany w rozdziale \ref{subsec:cele_pracy}.Można posłużyć się podsekcjami jak niżej. 

% Pierwszym było opracowanie bazy nagrań mowy ze stresem, na które składało się nagranie możliwie największej liczby osób w~rzeczywistej sytuacji stresującej, odpowiednia unifikacja parametrów technicznych nagrań, opracowanie specyfikacji technicznej bazy, utworzenie narzędzia umożliwiającego etykietowanie zbioru nagrań oraz opracowanie i~analiza podstawowej wersji bazy zawierającej etykiety. Drugim celem była implementacja modułu detekcji stresu oraz ewaluacja jego skuteczności na zebranej bazie.

\subsubsection*{Cel 1}

Pierwszy cel został osiągnięty, poprzez sporządzenie bazy nagrań....

\subsubsection*{Implementacja modułu analizy ...}

Drugi cel zeralizowano implementując....

\subsubsection*{Optymalizacja metody}
Po implementacji metody usprawniono działanie poprzez ....